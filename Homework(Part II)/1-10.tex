\documentclass[a4paper,12pt]{article}
\usepackage{ctex}
\usepackage{enumerate}
\usepackage{times}
\usepackage{mathptmx}
\usepackage{amsmath}
\usepackage{amsfonts}
\usepackage{amssymb}
\usepackage[all]{xy}
\usepackage[top=2cm, bottom=2cm, left=2cm, right=2cm]{geometry}

\DeclareMathOperator{\dom}{dom}
\DeclareMathOperator{\ran}{ran}
\DeclareMathAlphabet{\mathcal}{OMS}{cmsy}{m}{n}
\begin{document}
  \title{����~1-10~��ҵ}
  \author{��������ۿԴ \and ѧ�ţ�161240004}
  \date{}
  \maketitle

  \section{[UD] Problem 13.3}
  \begin{enumerate}[(a)]
    \item No. Because both $(1,\sqrt{3})$ and $(1,-\sqrt{3})$ are elements of $f$, however, $\sqrt{3} \neq -\sqrt{3}$.
    \item No. Because for $x=0$, there does not exist $y \in \mathbb{R}$, such that $y=1/(x+1)$.
    \item Yes. Because for all $(x,y) \in \mathbb{R}^2$, there exists a unique real number $z$ such that $z = x + y$.
    \item Yes. Because for every closed interval of real numbers $[a,b]$, there exists a unique real number $a$, such that $([a,b],a) \in f$.
    \item Yes. Because for every $(n,m) \in \mathbb{N} \times \mathbb{N}$, there exists a unique real number $m$, such that $((n,m),m) \in f$.
    \item Yes. Because for every real number $x$, there exists a real number $y$, such that $y=0$ when $x\geq 0$ or $y=x$ when $x < 0$, i.e. $(x,y) \in f$.
    \item No. Because both $(6,7)$ and $(6,5)$ are elements of $f$, however, $7 \neq 5$.
    \item Yes. Because for every circle $c$ in the plane $\mathbb{R}^2$, there exists a unique real number $C$, such that $C$ is the circumference of $c$.
    \item Yes. Because for every polynomial with real coefficients $p$, $p$ is differentiable, thus there exists a unique polynomial $p'$, such that $p'$ is the derivative of $p$.
    \item Yes. Because for every polynomial $p$, $p$ is integrable on $[0,1]$, thus there exists a unique number $I$ such that $I = \int_0^1 p(x)dx$.
  \end{enumerate}

  \section{[UD] Problem 13.4}
  We know that $A \cap \mathbb{N}$ is either an empty or a nonempty set. In the case that $A \cap \mathbb{N}$ is empty, there exists a unique integer $-1$, such that $(A,-1) \in f$. In the case that $A \cap \mathbb{N}$ is nonempty, $A \cap \mathbb{N}$ is a subset of $\mathbb{N}$. By well-ordering principle of $\mathbb{N}$, $\min(A\cap\mathbb{N})$ exists, so there exist a unique integer $\min(A\cap\mathbb{N})$, such that $(A,\min(A\cap\mathbb{N})) \in f$. Therefore $f$ is a well-defined function.

  \section{[UD] Problem 13.5}
  \begin{enumerate}[(a)]
    \item For all $x \in  X$, either $x \in A$ or $x \in X ~\backslash~ A$ holds, so there exists a unique number $y$ ($y=1$ when $x \in A$ and $y=0$ when $x \in X ~\backslash~ A$), such that $y = \chi_A$. Therefore $\chi_A$ is a function.
    \item The domain is $X$. The range is $\{0\}$ when $A = \varnothing$, $\{1\}$ when $A=X$, and $\{0,1\}$ when $A \neq \varnothing$ and $A \neq X$.
  \end{enumerate}

  \section{[UD] Problem 13.7}
    For every real number $y \neq 1/2$, let $(x-5)/(2x-3)=y$, and we get $x = (3y-5)/(2y-1) \neq 3/2$, which is an element of the domain. So $\ran(f) = \mathbb{R} ~\backslash~\{1/2\}$. \hfill $\square$

  \section{[UD] Problem 13.11}
  No. For every $x \in A$, there may not exist $y$ such that $(x,y) \in f$. Even though for every $x \in A$ there exists $y$ such that $(x,y) \in f$, we cannot make sure that there only exists one $y$ such that $(x,y) \in f$.

  \section{[UD] Problem 13.13}
  The only relation is $\{(x,y) \in X^2:x=y\}$. By the reflexion of the equivalence, any relation on $X$ is superset of $\{(x,y) \in X^2:x=y\}$. Assume there exists relation $X'$ such that $X' ~\backslash~ X \neq \varnothing$, let $(a,b)$ be an element of $X'$ such that $a \neq b$. However, $(b,b)$ is an element of $X'$ but $a \neq b$, so $X'$ is not a function.

  \section{[UD] Problem 14.8}
  \begin{enumerate}[(a)]
    \item Not one-to-one. $f(1)=f(-1)=1/2$ but $1 \neq -1$.\\
      Not onto. The range is $(0,1]$.
    \item Not one-to-one. $\sin 0 = \sin \pi = 0$ but $0 \neq \pi$.\\
      Not onto. The range is $[-1,1]$.
    \item Not one-to-one. $f(1,2)=f(2,1)=2$ but $(1,2) \neq (2,1)$.\\
        Onto.
    \item Not one-to-one. $f((1,0),(0,0))=f((0,0),(0,0))=0$ but $((1,0),(0,0)) \neq ((0,0),(0,0))$. \\
        Onto.
    \item Not one-to-one. $f((0,0),(0,0))=f((1,1),(1,1))=0$ but $((0,0),(0,0)) \neq ((1,1),(1,1))$. \\
      Not onto. The range is $[0,+\infty)$.
    \item One-to-one. \\
      Not onto. The range is $A \times \{b\}$.
    \item One-to-one.\\
    Onto.
    \item Not one-to-one. $f(X)=f(B)=B$ but $X \neq B$. \\
      Not onto. The range is $\mathcal{P}(X ~\backslash~ B)$.
    \item One-to-one.\\
      Not onto. The range is $(0,+\infty)$.
  \end{enumerate}

  \section{[UD] Problem 14.12}
  $f(x)=\dfrac{(d-c)x+cb-da}{b-a} (x \in [a,b])$. \par
  One-to-one: Let $f(x_1)=f(x_2)$, we have $\dfrac{(d-c)x_1+cb-da}{b-a} = \dfrac{(d-c)x_2+cb-da}{b-a}$. Multiplying $b-a$ and cancelling on both sides, we have $x_1=x_2$. \par
  Onto: Let $ c \leq f(x) \leq d$, that is $c \leq \dfrac{(d-c)x+cb-da}{b-a} \leq d$. Multiplying $b-a$ and cancelling on both sides, we have $a \leq x \leq b$. It means, for every $x \in [a,b]$, there exists $y$, such that $y=f(x)$, thus $f(x)$ is onto. \par
  Since $f(x)$ is both one-to-one and onto, $f(x)$ is a bijection. \hfill $\square$

  \section{[UD] Problem 14.13}
  $\phi$ is a function from $F([0,1])$ to $\mathbb{R}$. Because for all $f \in F([0,1])$, there exists a unique real number $y$, such that $y = f(0)$. \par
  $\phi$ is not one-to-one. Let $f_1(x)=0 \in F([0,1])$,  $f_2(x)=x \in F([0,1])$, we have that $\phi(f_1)=\phi(f_2)$, however, $f_1\neq f_2$ because $f_1(1) \neq f_2(1)$. \par
  $\phi$ is onto. For every real number $a$, there exists $f_0(x)=a \in F([0,1])$, such that  $\phi(f_0) = a$.

  \section{[UD] Problem 14.15}
  For all $x \in \mathbb{R}$, since $f(x)$ is defined on $\mathbb{R}$, there exists a unique real number $y=f(x) \cdot f(x)$, such that $y = (f \cdot f)(x)$, therefore $f \cdot f$ is a function. \hfill $\square$
  \begin{enumerate}[(a)]
    \item Yes. $f(x) = e^x$.
    \item No. $\ran(f \cdot f) = \{a^2: a \in \ran(f)\}$.
  \end{enumerate}

  \section{[UD] Problem 15.1}
  \begin{tabular}{|c|c|c|c|c|c|c|}
    \hline
     & $(f \circ g)(x)$ & $\dom(f \circ g)$ & $\ran(f \circ g)$ & $(g \circ f)(x)$ & $ \dom(g \circ f)$ & $\ran(g \circ f)$ \\
    \hline
    (a) & $1/(1+x^2)$ & $\mathbb{R}$ & $(0,1]$ & $1/(1+x)^2$ & $\mathbb{R} ~\backslash~ \{-1\}$ & $\mathbb{R}^+$ \\
    \hline
    (b) & $x$ & $\mathbb{R}^+$ & $\mathbb{R}^+$ & $|x|$ & $\mathbb{R}$ & $[0,+\infty)$ \\
    \hline
    (c) & $1/(x^2+1)$ & $\mathbb{R}$ & $(0,1]$ & $(1/x^2)+1$ & $\mathbb{R} ~\backslash~ \{0\}$ & $(1,+\infty)$  \\
    \hline
    (d) & $|x|$ & $\mathbb{R}$ & $[0,+\infty)$ & $|x|$ & $\mathbb{R}$ & $[0,+\infty)$ \\
    \hline
  \end{tabular}

  \section{[UD] Problem 15.6}
  \begin{enumerate}[(a)]
    \item $ (f \circ g) (x) = f(g(x))
      = \dfrac{\dfrac{3+2x}{1-x}-3}{\dfrac{3+2x}{1-x}+2}
      = \dfrac{\dfrac{5x}{1-x}}{\dfrac{5}{1-x}}
      = x $, \\ [0.2cm]
      $ (g \circ f) (x) = g(f(x))
      = \dfrac{3+2\dfrac{x-3}{x+2}}{1-\dfrac{x-3}{x+2}}
      = \dfrac{\dfrac{5x}{x+2}}{\dfrac{5}{x+2}}
      = x $.
    \item (Theorem 15.4) Let $f: A \rightarrow B$ be a bijective function, and $f^{-1}$ be the inverse of $f$, then $ f \circ g  = i_B$, and $ g \circ f  = i_A$.
  \end{enumerate}

  \section{[UD] Problem 15.7}
  \begin{enumerate}[(a)]
    \item
    \begin{enumerate}[(i)]
      \item $f=\{(1,4),(2,5),(3,5)\}, \;\; g=\{(4,1),(5,2)\}$;
      \item $f=\{(1,4),(2,5)\}, \;\; g=\{(4,1),(5,2)\}$;
      \item Impossible.
    \end{enumerate}
    \begin{center}
         \begin{tabular}{ccc}
               \xymatrix@R=0.2cm{
        \mathbf{B} \quad \ar[r]^{g} & \quad \mathbf{A} \quad \ar[r]^{f} & \quad \mathbf{B} \\
        4 \quad \ar[r] & \quad 1 \quad \ar[r] & \quad 4 \\
        5 \quad \ar[r] & \quad 2 \quad \ar[r] & \quad 5 \\
         & \quad 3 \ar[ur]  \quad &    } &  \hspace{1cm} &  \xymatrix@R=0.2cm{
        \mathbf{B} \quad \ar[r]^{g} & \quad \mathbf{A} \quad \ar[r]^{f} & \quad \mathbf{B} \\
        4 \quad \ar[r] & \quad 1 \quad \ar[r] & \quad 4 \\
        5 \quad \ar[r] & \quad 2 \quad \ar[r] & \quad 5 \\
         &  \quad &    } \\
           (i) &  &(ii) \\
         \end{tabular} \par
         Figure 1: diagrams of $A$ and $B$
    \end{center}
    \item Let $A=\{1,2\}$, $B=\{1\}$, $f=\{(1,1),(2,1)\}$, $g=\{(1,1)\}$, we have $f \circ g = \{(1,1)\} = i_B$, but $g \circ f = \{(1,1),(2,1)\} \neq i_A$. \par
        Because neither $f$ nor $g$ is a bijective function.
    \item Let $A=\{1\}$, $B=\{1,2\}$, $f=\{(1,1)\}$, $g=\{(1,1),(2,1)\}$, we have $g \circ f = \{(1,1)\} = i_A$, but $f \circ g = \{(1,1),(2,1)\} \neq i_B$. \par
        Because neither $f$ nor $g$ is a bijective function.
    \item $f$ is not always one-to-one, but must be onto. For injectivity, we have a counterexample in (b). For surjectivity, suppose to the contrary that $f$ is not onto.
        That means, there exists $b \in B$, for all $a \in A$, $f(a) \neq b$. Therefore, $(f \circ g)(b) = f(g(b)) \neq b$, which is contradict to that $f \circ g = i_B$. Therefore $f$ is onto.
    \item Guess whether the function has some property. If true, try to find the proof; if false, try to find a counterexample. \par
        Here, $f$ is not always onto, but must be one-to-one. For surjectivity, we have a counterexample in (c). For injectivity, suppose to the contrary that $f$ is not one-to-one. That means, there exists $a$ and $b$ in $A$ such that $f(a)=f(b)$. However, $(g \circ f)(a)=g(f(a))=g(f(b))=(g \circ b)(b)$, which is contradict to that $g \circ f = i_A$. Therefore $f$ is one-to-one.
  \end{enumerate}

  \section{[UD] Problem 15.11}
  By the definition of the inverse of a function, the inverse function of $f$ exists because $f$ is a bijection. Since $f \circ g_1 = f \circ g_2$, we have $f^{-1} \circ (f \circ g_1) = f^{-1} \circ (f \circ g_2)$, thus $(f^{-1} \circ f) \circ g_1 = (f^{-1} \circ f) \circ g_2$ because the composition satisfies associative property, and by Theorem 15.4 (ii) we get $g_1 = g_2$.  \hfill $\square$ \par
  If $g_1 \circ f = g_2 \circ f$ and $f$ is bijective, $g_1=g_2$ still holds. Just get $g_1 \circ (f \circ f^{-1}) = g_2 \circ (f \circ f^{-1})$, and prove in the similar way.

  \section{[UD] Problem 15.12}
  Yes. \par
  The equivalence class of $a \in A$ is $\{x: f(x)=f(a)\}$.

  \section{[UD] Problem 15.13}
  No. \par
  Yes. $f(x)=x$.

  \section{[UD] Problem 15.14}
  \begin{enumerate}[(a)]
    \item First, for all $(a,c) \in A \times C$, there exists a unique pair $(f(a),g(c)) \in B \times D$, such that $H(a,c) = (f(a),g(c))$ because $f: A \rightarrow B$ and $g:
        C \rightarrow D$ are both functions. Therefore $H$ is a function. \par
        Second, let $(f(a_1),g(c_1)) = (f(a_2),g(c_2))$, by the definition of ordered pair, we have $f(a_1) = f(a_2)$ and $g(c_1)=g(c_2)$, since $f$ and $g$ are both one-to-one, we get $a_1 = a_2$ and $c_1=c_2$, and this implies $(a_1,c_1)=(a_2,c_2)$. Therefore $H$ is one-to-one. \hfill $\square$
    \item Since $f$ and $g$ are onto, for every $(b,d) \in B \times D$, there exist $a$ and $c$, such that $f(a)=b$ and $g(c)=d$, therefore $H(a,c)=(b,d)$. Hence $H$ is also onto. \hfill $\square$
  \end{enumerate}

  \section{[UD] Problem 15.15}
  $H$ is not a function: $A=\{1,2\}$, $B=\{1,2\}$, $C=\{2,3\}$, $D=\{3,4\}$, $f=\{(1,1),(2,2)\}$, $g=\{(2,3),(3,4)\}$, $H=\{(1,1),(2,2),(2,3),(3,4)\}$. \par
  $H$ is a function: $A=\{1\}$, $B=\{1\}$, $C=\{2\}$, $D=\{2\}$, $f=\{(1,1)\}$, $g=\{(2,2)\}$, $H=\{(1,1),(2,2)\}$. \par
  When $A$ and $C$ are disjoint, we are assured that $H$ is a function. In fact, $H$ is a function if and only if $f \cap [(A \cap C) \times B] = g \cap [(A \cap C) \times D]$.
  
  \section{[UD] Problem 15.20}
  \begin{enumerate}[(a)]
    \item Let $f|_{A_1}(x)=f|_{A_1}(y)$, and by the definition of the restriction function, we have $f(x)=f(y)$. Since $f$ is one-to-one, we have $x=y$. Therefore $f|_{A_1}$ is one-to-one. \hfill $\square$
    \item For every $y \in B$, there exists $x \in A_1 \subset A$ such that $f|_{A_1}(x)=f(x)=y$ because $f|_{A_1}$ is onto. Therefore $f$ is onto. \hfill $\square$
  \end{enumerate}

  \section{[UD] Problem 16.19}
  For every $a \in A$, there exists $b \in B$ such that $b = f(a)$ , and we have that $f^{-1}(\{b\}) \subseteq A$ because $f$ is a function from $A$ to $B$. Therefore, $\bigcup \limits_{b \in B} f^{-1}(\{b\}) = A$. \par
  Since $f$ is onto, for every $b \in B$, there exists $a \in A$, such that $f(a)=b$, thus $f^{-1}(\{b\})$ is always nonempty. \par
  If $f^{-1}(\{b_1\}) \cap f^{-1}(\{b_2\})$ is nonempty, there exists $a$, such that $f(a)=b_1$ and $f(a)=b_2$, thus $b_1=b_2$, therefore $f^{-1}(\{b_1\}) = f^{-1}(\{b_2\})$. \par
  Summarizing, we conclude that $\{f^{-1}(\{b\}):b \in B\}$ is a partition of $A$. \hfill $\square$

  \section{[UD] Problem 16.20}
  \begin{enumerate}[(a)]
    \item No.
    \item For every $a \in A_1$, we have $f(a) \in f(A_1) = f(A_2)$, thus there exists $a' \in A_2$ such that $f(a')=f(a)$. Since $f$ is \textbf{one-to-one}, we have that $a'=a$, therefore $a \in A_2$. Hence $A_1 \subseteq A_2$, and $A_2 \subseteq A_1$ likewise. Therefore $A_1=A_2$.  \hfill $\square$ \par
        I used only one-to-one.
  \end{enumerate}

  \section{[UD] Problem 16.21}
  \begin{enumerate}[(a)]
    \item No.
    \item For every $b \in B_1 \subseteq Y$, there exists $a \in X$ such that $f(a)=b$ because $f$ is \textbf{onto}. Hence, $a$ is an element of $f^{-1}(B_1) = f^{-1}(B_2)$, therefore there exists $b' \in B_2$ such that $f(a)=b'$, thus $b=b'$, and $b$ is an element of $B_2$. Therefore $B_1$ is a subset of $B_2$, and $B_2$ is a subset of $B_1$ likewise. So $B_1=B_2$. \hfill $\square$ \par
        I used only onto.
  \end{enumerate}

  \section{[UD] Problem 16.22}
  \begin{enumerate}[(a)]
    \item Yes.
    \item For all $x \in A_1 \cap A_2$, both $\chi_{A_1}(x)$ and $\chi_{A_2}(x) = 1$, therefore $\chi_{A_1 \cap A_2}(x)= \chi_{A_1}(x) \cdot \chi_{A_2}(x) = 1$. \par
         For all $x \notin A_1 \cap A_2$, either $\chi_{A_1}(x)$ or $\chi_{A_2}(x) = 0$, therefore $\chi_{A_1 \cap A_2}(x)= \chi_{A_1}(x) \cdot \chi_{A_2}(x) = 0$. \par
         Summarizing, we have $\chi_{A_1} \cdot \chi_{A_2} = \chi_{A_1 \cap A_2}$. \hfill $\square$ 
    \item For all $x$ s.t. $x \in A_1$ and $x \in A_2$, $\chi_{A_1}(x) = \chi_{A_2}(x) = 1$, $\chi_{A_1 \cap A_2}(x) = 1$, therefore $\chi_{A_1 \cup A_2}(x)= \chi_{A_1}(x) + \chi_{A_2}(x) - \chi_{A_1 \cap A_2}(x) = 1$ \par
        For all $x$ s.t. $x \in A_1$ and $x \notin A_2$, $\chi_{A_1}(x) = 1$,$\chi_{A_2}(x) = 0$, $\chi_{A_1 \cap A_2}(x) = 0$, therefore $\chi_{A_1 \cup A_2}(x)= \chi_{A_1}(x) + \chi_{A_2}(x) - \chi_{A_1 \cap A_2}(x) = 1$ \par
        For all $x$ s.t. $x \notin A_1$ and $x \in A_2$, $\chi_{A_1 \cup A_2}(x)= \chi_{A_1}(x) + \chi_{A_2}(x) - \chi_{A_1 \cap A_2}(x) = 1$ holds likewise. \par
        For all $x$ s.t. $x \notin A_1$ and $x \notin A_2$, $\chi_{A_1}(x) = \chi_{A_2}(x) = 0$, $\chi_{A_1 \cap A_2}(x) = 0$, therefore $\chi_{A_1 \cup A_2}(x)= \chi_{A_1}(x) + \chi_{A_2}(x) - \chi_{A_1 \cap A_2}(x) = 0$ \par
        Summarizing, we have $\chi_{A_1 \cup A_2} = \chi_{A_1} + \chi_{A_2} - \chi_{A_1 \cap A_2}$. \hfill $\square$ 
    \item $\chi_{X ~\backslash~ A_1} = 1 - \chi_{A_1}$.
  \end{enumerate}
  
\end{document}
