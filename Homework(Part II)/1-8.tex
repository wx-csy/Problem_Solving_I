\documentclass[a4paper,11pt]{article}
\usepackage{ctex}
\usepackage{enumerate}
\usepackage{times}
\usepackage{mathptmx}
\usepackage{amsmath}
\usepackage{amsfonts}
\usepackage{amssymb}
\usepackage[top=2cm, bottom=2cm, left=2cm, right=2cm]{geometry}

\DeclareMathAlphabet{\mathcal}{OMS}{cmsy}{m}{n}
\begin{document}
  \title{����~1-8~��ҵ}
  \author{��������ۿԴ \and ѧ�ţ�161240004}
  \date{}
  \maketitle

  \section{[UD] Problem 6.7}
  \begin{itemize}
    \item $B ~\backslash~ A$;
    \item $(A \cup B) ~\backslash~ (A \cap B)$;
    \item $A \cap B \cap C$;
    \item $(B \cap C) ~\backslash~ A$;
    \item $((A \cap B) \cup (A \cap C) \cup (B \cap C)) ~\backslash~ (A \cap B \cap C)$.
  \end{itemize}

  \section{[UD] Problem 6.16}
  \begin{enumerate}[(a)]
    \item For every $n$ in $A$, $n = x^2$ where $x$ is an integer, therefore $n$ is an integer, i.e. $n \in B$, so $A \subseteq B$. \hfill $\square$
    \item For every $t$ in $A$, $t$ is a real number, there exists a real number $x=t/2$, such that $t=2x$, so $t \in B$. Therefore $A \subseteq B$. \hfill $\square$
    \item For every point $(x,y)$ in $A$, we have $y=(5-3x)/2$, therefore $2y+3x=5$, which means that $(x,y)$ is also an element of $B$. So $A \subseteq B$. \hfill $\square$
  \end{enumerate}

  \section{[UD] Problem 6.17}
  \begin{enumerate}[(a)]
    \item $A$ is a proper subset of $B$. For every $(x,y)$ in $A$, we have $xy>0$, so both $x$ and $y$ are nonzero, thus $x^2+y^2>0$, therefore $A$ is a subset of $B$. However, $(1,-1)$ is an element of $B$, but not an element of $A$, so $A$ is a proper subset of $B$. \hfill $\square$
    \item $A$ is a proper subset of $B$. By theorem 6.10, we have $A \subseteq B$. However, $(0,0)$ is an element of $B$, but not an element of $A$, so $A$ is a proper subset of $B$. \hfill $\square$
  \end{enumerate}

  \section{[UD] Problem 7.1}
  \begin{enumerate}[(a)]
    \item For every $x$ in universe, by definition of complement, if $x \in A$, then $x \notin A^c$ and if $x \notin A^c$ then $x \in (A^c)^c$, therefore we have if $x \in A$, then $x \in (A^c)^c$, i.e. $A$ is a subset of $(A^c)^c$. $(A^c)^c$ is a subset of $A$ likewise. So $(A^c)^c=A$. \hfill $\square$
    \item For every $x$ in $A \cap (B \cup C)$, we have $x \in A$, and $x \in B$ or $C$, so $x \in A$ and $B$ or $x \in A$ and $C$, thus $A \cap (B \cup C)$ is a subset of $(A \cap B) \cup (A \cap C)$. For every $x$ in $(A \cap B) \cup (A \cap C)$, we have $x \in A$ and $B$ or $x \in A$ and $C$, so $x \in A$, and $x \in B$ or $C$, thus $(A \cap B) \cup (A \cap C)$ is a subset of $A \cap (B \cup C)$. So $A \cap (B \cup C) = (A \cap B) \cup (A \cap C)$. \hfill $\square$
    \item For every $x$ in $X ~\backslash~ (A \cap B)$, we have $x \in X$ and, $x \notin A$ or $x \notin B$, thus $x \in X$ and $x \notin A$, or $x \in X$ and $x \notin B$, therefore $X ~\backslash~ (A \cap B)$ is a subset of $(X ~\backslash~ A) \cup (X ~\backslash~ B)$. For every $x$ in $(X ~\backslash~ A) \cup (X ~\backslash~ B)$, we have $x \in X$ and $x \notin A$, or $x \in X$ and $x \notin B$, thus $x \in X$ and, $x \notin A$ or $x \notin B$, so $(X ~\backslash~ A) \cup (X ~\backslash~ B)$ is a subset of $X ~\backslash~ (A \cap B)$. Therefore $X ~\backslash~ (A \cap B) = (X ~\backslash~ A) \cup (X ~\backslash~ B)$. \hfill $\square$
    \item Since $A, B$ are subsets of $X$, for every $x \in X$, ``if $x \in A$ then $x \in B$'' and ``if $x \notin B$ then $x \notin A$'' are equivalent, so $A \subseteq B$ if and only if $(X ~\backslash~ B) \subseteq (X ~\backslash~ A)$. \hfill $\square$
    \item If $A \cap B = B$, then for every $x$, ``$x \in B$'' and ``$x \in A$ and $B$'' are equivalent, so $x \in B$ implies $x \in A$, i.e. $A$ is a subset of $B$. If $B \subseteq A$, for every $x$, $x \in B$ implies $x \in A$, thus $x \in B$ and $x \in A$ and $B$ are equivalent, so $A \cap B = B$. Therefore, $A \cap B = B$ if and only if $B \subseteq A$. \hfill $\square$
  \end{enumerate}

  \section{[UD] Problem 7.8}
  \begin{enumerate}[(a)]
    \item (ii);
    \item (i), (ii), (iii), (iv), (v);
    \item For every $x$ in $(A \cap B)~\backslash~ C$, we have $x \in A, B$ and $x \notin C$, so $x \in A$ and $x \notin C$, and $x \in B$ and $x \notin C$, thus $(A \cap B)~\backslash~ C$ is a subset of $(A~\backslash~ C) \cap ( B~\backslash~ C)$. Likewise $(A~\backslash~ C) \cap ( B~\backslash~ C)$ is a subset of $(A \cap B)~\backslash~ C$. Therefore $(A \cap B)~\backslash~ C = (A~\backslash~ C) \cap ( B~\backslash~ C)$. \hfill $\square$

  \end{enumerate}

  \section{[UD] Problem 7.9}
  \begin{enumerate}[(a)]
    \item For every $x$ in $A ~\backslash~ B$, we have $x \in A$ and $x \notin B$, so $A~\backslash~B$ and $B$ are disjoint. \hfill $\square$
    \item For every $x$ in $A \cup B$, we have $x \in A$ or $x \in B$, so $x \in A$, or $x \in B$ and $x \notin A$, therefore $A \cup B$ is a subset of $(A ~\backslash~ B) \cup B$. For every $x$ in $(A ~\backslash~ B) \cup B$, we have $x \in A$, or $x \in B$ and $x \notin A$, so $x \in A$ or $x \in B$, therefore $(A ~\backslash~ B) \cup B$ is a subset of $A \cup B$. So $A \cup B = (A ~\backslash~ B) \cup B$ \hfill $\square$
  \end{enumerate}

  \section{[UD] Problem 7.10}
    This statement is false. Here is a counterexample. Let $A=\{1,2\}$, $B=\{1\}$ and $C=\{2\}$, then $A \cup B = A \cup C$, but $B \neq C$. \hfill $\square$

  \section{[UD] Problem 7.11}
    This statement is true. We know that for every $x$, $x \in S$ if and only if $\{x\} \cap S = \{x\}$. For every $x \in A$, let $Y = \{x\}$, then $B \cap Y = A \cap Y = \{x\}$, so $x \in B$, thus $A$ is a subset of $B$. $B$ is a subset of $A$ likewise. So the statement is true.  \hfill $\square$

  \section{[UD] Problem 8.1}
  \begin{enumerate}[(a)]
    \item $\displaystyle \bigcup_{n=1}^\infty A_n=[0, 1) \cup [0,1/2) \cup [0,1/3) \cdots = [0,1)$ ; \\
        $\displaystyle \bigcup_{n=1}^\infty B_n=[0, 1] \cup [0,1/2] \cup [0,1/3] \cdots = [0,1]$ ; \\
        $\displaystyle \bigcup_{n=1}^\infty C_n=(0, 1) \cup (0,1/2) \cup (0,1/3) \cdots = (0,1)$
    \item $\displaystyle \bigcap_{n=1}^\infty A_n=[0, 1) \cap [0,1/2) \cap [0,1/3) \cdots = \{0\}$ ; \\
        $\displaystyle \bigcap_{n=1}^\infty B_n=[0, 1] \cap [0,1/2] \cap [0,1/3] \cdots = \{0\}$ ; \\
        $\displaystyle \bigcap_{n=1}^\infty C_n=(0, 1) \cap (0,1/2) \cap (0,1/3) \cdots = \varnothing$
    \item No. Because $A_0$ is undefined.
  \end{enumerate}

  \section{[UD] Problem 8.4}
  This statement is false. Here is a counterexample: let $A_n=(n,n+\dfrac{1}{2})$, $B_n=[n,n+\dfrac{1}{2}]$, for all positive integer $n$, $A_n \subset B_n$ holds, however,
    $$\bigcap_{n=1}^\infty A_n = \bigcap_{n=1}^\infty B_n = \varnothing$$
  which does not satisfy the definition of strict inclusion.

  \section{[UD] Problem 8.7}
  \begin{enumerate}[(a)]
    \item Suppose, to the contrary, that $\displaystyle \bigcap_{\alpha \in I} A_\alpha \neq \varnothing$, then there exists $\displaystyle x \in \bigcap_{\alpha \in I} A_\alpha$, however, there exists $\alpha_0 \in I$ such that $A_{\alpha_0} = \varnothing$, so $x \in \varnothing$, which leads to a contradiction. Therefore $\displaystyle \bigcap_{\alpha \in I} A_\alpha = \varnothing$. \hfill $\square$
    \item Let $X$ be the universe. For all $x$ in $X$, there exists $\alpha_0 \in I$, such that $A_{\alpha_0}=X$, thus $x \in X$, so $\displaystyle x \in \bigcup_{\alpha \in I} A_\alpha = X$. \hfill $\square$
    \item For all $x \in B$, for all $\alpha \in I$, we have $x \in A_\alpha$, so $\displaystyle x \in \bigcap_{\alpha \in I} A_\alpha$, therefore $\displaystyle B \subseteq \bigcap_{\alpha \in I} A_\alpha$.  \hfill $\square$
  \end{enumerate}

  \section{[UD] Problem 8.8}
  $A = \mathbb{Z}$. \par
  Proof: for every integer $m$, there exists $n=|m|$ such that $m \notin \mathbb{R} ~\backslash~ \{-n,-n+1,\cdots,0,\cdots,n-1,n\}$, thus $\displaystyle m \notin \bigcap_{n \in \mathbb{Z}^+} \mathbb{R} ~\backslash~ \{-n,-n+1,\cdots,0,\cdots,n-1,n\}$, i.e. $m \in A$, so $\mathbb{Z}$ is a subset of $A$. For every $x$ in $A$, $\displaystyle x \notin \bigcap_{n \in \mathbb{Z}^+} \mathbb{R} ~\backslash~ \{-n,-n+1,\cdots,0,\cdots,n-1,n\}$, that is, there exists a positive integer $n$, such that $x \notin \mathbb{R} ~\backslash~ \{-n,-n+1,\cdots,0,\cdots,n-1,n\}$, i.e. $x \in \{-n,-n+1,\cdots,0,\cdots,n-1,n\}$, so $x \in \mathbb{Z}$, therefore $\mathbb{Z}$ is a subset of $A$. By the definition of equality of two sets, we get $\mathbb{Z} = A$. \hfill $\square$

  \section{[UD] Problem 8.9}
  $A= \{n: n = 2m, m \in \mathbb{Z}\}$. \par
  Proof: let $\mathbb{R}$ be the universe,
  \begin{align*}
    A &= \mathbb{Q} ~\backslash~ \bigcap_{n \in \mathbb{Z}}(\mathbb{R}~\backslash~\{2n\}) \\
      &= \mathbb{Q} ~\backslash~ (\mathbb{R} ~\backslash~ \bigcup_{n \in \mathbb{Z}}\{2n\}) & \text{(By De Morgan's laws)} \\
      &= \mathbb{Q} ~\backslash~ (\bigcup_{n \in \mathbb{Z}}\{2n\})^c & \text{(By the definition of complement)} \\
      &= \mathbb{Q} ~\backslash~ (\{n: n=2m, m \in \mathbb{Z}\}) ^c \\
      &= \mathbb{Q} \cap (\{n: n=2m, m \in \mathbb{Z}\} ^c ) ^c & \text{(By Theorem 7.4.17)} \\
      &= \mathbb{Q} \cap \{n: n=2m, m \in \mathbb{Z}\} & \text{(By Theorem 7.4.2)} \\
      &= \{n: n=2m, m \in \mathbb{Z}\}  & \square
  \end{align*}

  \section{[UD] Problem 8.11}
  \begin{enumerate}[(a)]
    \item $A_\alpha=\{\alpha\}$, $(\alpha \in \mathbb{Z})$;
    \item if $A_\alpha \neq A_\beta$, then $A_\alpha \cap A_\beta = \varnothing$;
    \item if $A_\alpha = A_\beta$, then $A_\alpha \cap A_\beta \neq \varnothing$;
    \item Yes.
    \item Yes.
    \item This assertion holds if and only if there is more than one element in $I$.
    \item No. Here is a counterexample: $\{A_\alpha : A_\alpha = \{1,2,3\} ~\backslash~ \{\alpha\}, \alpha \in I\}$ $(I=\{1,2,3\})$.
  \end{enumerate}

  \section{[UD] Problem 9.2}
  \begin{enumerate}[(a)]
    \item By the definition of the power set, for every $X$ in  $\mathcal{P}(A) \cup \mathcal{P}(A)$, $X$ is a subset of $A$ or $B$, so $X$ is a subset of $A \cup B$, therefore $\mathcal{P}(A) \cup \mathcal{P}(A) \subseteq \mathcal{P}(A \cup B)$.
    \item Let $A=\{1\}$, $B=\{2\}$, then $A \cup B = {1,2}$, $\mathcal{P}(A)=\{\{1\},\varnothing\}$, $\mathcal{P}(B)=\{\{2\},\varnothing\}$,
        $\mathcal{P}(A) \cup \mathcal{P}(B) = \{\{1\},\{2\},\varnothing\}$, $\mathcal{P}(A \cup B)=\{\{1\},\{2\},\{1,2\},\varnothing\}$, $\mathcal{P}(A) \cup \mathcal{P}(B) \neq \mathcal{P}(A \cup B)$.
  \end{enumerate}

  \section{[UD] Problem 9.4}
  If $A \subseteq B$, then for all set $X$ such that $X$ is a subset of $A$, $X$ is also a subset of $B$, so $\mathcal{P}(A) \subseteq \mathcal{P}(B)$. \par
  If $\mathcal{P}(A) \subseteq \mathcal{P}(B)$, then $A \in \mathcal{P}(A) \subseteq \mathcal{P}(B)$, thus $A$ is a subset of $B$. \par
  Therefore, $A \subseteq B$ if and only if $\mathcal{P}(A) \subseteq \mathcal{P}(B)$. \hfill $\square$

  \section{[UD] Problem 9.12}
  \begin{enumerate}[(a)]
    \item The sufficiency is obvious, and we only have to prove the necessity. Assume, to the contrary, that $A \neq C$ or $B \neq D$. Without loss of generality, suppose $A \neq C$. Therefore, there exists $x$ such that $x \in A$ and $x \notin C$, or $x \notin A$ and $x \in C$. Suppose $x \in A$ and $x \notin C$ without loss of generality. Since $B$ is a nonempty set, there exists $y$ in $B$. Consider $(x,y)$, by the definition of Cartesian product, it is an element of $A \times B$, however it isn't an element of $C \times D$, because $x$ is not an element of $C$. Therefore, $A \times B = C \times D$ if and only if $A = C$ and $B = D$. \hfill $\square$
    \item When constructing the pair $(x,y)$ (which leads to contradiction), we need to take an element $y$ in $B$, and this requires the sets are nonempty.
  \end{enumerate}

  \section{[UD] Problem 9.13}
  No. Let $A=\{1\}, C=\{2\}, B=D=\varnothing$, then $A \times B = C \times D = \varnothing$, so $A \times B \subseteq C \times D$, however $A \nsubseteq C$.

  \section{[UD] Problem 9.14}
  \begin{enumerate}[(a)]
    \item True. For all $(x,y)$ in $A \times (B \cup C)$, we have $x \in A$ and $y \in B \cup C$, so $x \in A$ and $y \in B$, or $x \in A$ and $y \in C$, hence $x \in (A \times B) \cup (A \times C)$, therefore, $A \times (B \cup C)$ is a subset of $x \in (A \times B) \cup (A \times C)$. For all $(x,y)$ in $x \in (A \times B) \cup (A \times C)$, we have $x \in A$ and $y \in B$, or $x \in A$ and $y \in C$, so $x \in A$ and $y \in B \cup C$, therefore $x \in (A \times B) \cup (A \times C)$ is a subset of $A \times (B \cup C)$. By the definition of the equality of two sets, we get $A \times (B \cup C)=x \in (A \times B) \cup (A \times C)$. \hfill $\square$
    \item True. For all $(x,y)$ in $A \times (B \cap C)$, we have $x \in A$ and $y \in B \cap C$, so $x \in A$ and $y \in B$, and $x \in A$ and $y \in C$, hence $x \in (A \times B) \cap (A \times C)$, therefore, $A \times (B \cap C)$ is a subset of $x \in (A \times B) \cap (A \times C)$. For all $(x,y)$ in $x \in (A \times B) \cap (A \times C)$, we have $x \in A$ and $y \in B$, and $x \in A$ and $y \in C$, so $x \in A$ and $y \in B \cap C$, therefore $x \in (A \times B) \cap (A \times C)$ is a subset of $A \times (B \cap C)$. By the definition of the equality of two sets, we get $A \times (B \cap C)=x \in (A \times B) \cap (A \times C)$. \hfill $\square$
  \end{enumerate}

  \section{[UD] Problem 9.16}
  \begin{enumerate}[(a)]
    \item If $a=b$, then $(a,b)=\{\{a\},\{a,b\}\}=\{\{a\},\{a\}\}=\{\{a\}\}$, thus $\{\{x\},\{x,y\}\}=\{\{a\}\}$, so $\{x\}=\{x,y\}=\{a\}$, therefore $x=y=a$. Hence $a=x$ and $b=y$. \par
        If $a \neq b$, then $\{\{a\},\{a,b\}\}=\{\{x\},\{x,y\}\}$. Suppose $\{a\}=\{x,y\}$, then $a=x=y$, so $\{\{a\},\{a,b\}\}=\{\{x\},\{x,y\}\}=\{\{x\}\}$, thus $\{a\}=\{a,b\}$, and therefore $a=b$, which contradicts $a \neq b$. So $\{a\}=\{x\}$, therefore $a=x$ and $\{a,b\}=\{x,y\}$. If $a=y$, then $a=x=y$, thus $\{a,b\}=\{x,y\}=\{x\}$, therefore $a=b$, which contradicts $a \neq b$. So $a \neq y$, therefore $a=x$ and $b=y$. \hfill $\square$
    \item Since $\{a\}$ and $\{a,b\}$ are subsets of $A \cup B$, they are both the elements of $\mathcal{P}(A \cup B)$,  therefore, $(a,b) = \{\{a\},\{a,b\}\} \in \mathcal{P}(\mathcal{P}(A \cup B))$. \hfill $\square$
    \item For all $x \in \mathcal{P}(\mathcal{P}(A \cup B))$ in $A \times B$, there exists $a \in A$ and $b \in B$, such that $x = (a,b)$. Therefore, $x \in \mathcal{P}(\mathcal{P}(C \cup D))$ (apply the conclusion in Problem 9.4 twice), and $a \in C$ and $b \in D$, so $x \in C \times D$. Therefore, $A \times B \subseteq C\times D$. \hfill $\square$
  \end{enumerate}

\end{document}
