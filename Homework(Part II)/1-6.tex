\documentclass[a4paper,12pt]{article}
\usepackage{ctex}
\usepackage{enumerate}
\usepackage{times}
\usepackage{mathptmx}
\usepackage{amsmath}
\usepackage{amsfonts}
\usepackage[top=2cm, bottom=2cm, left=2cm, right=2cm]{geometry}
\begin{document}
  \title{����~1-6~��ҵ}
  \author{��������ۿԴ \and ѧ�ţ�161240004}
  \date{}
  \maketitle

  \section{[UD] Problem 6.7}
  \begin{itemize}
    \item $B ~\backslash~ A$;
    \item $(A \cup B) ~\backslash~ (A \cap B)$;
    \item $A \cup B \cup C$;
    \item $(B \cap C) ~\backslash~ A$;
    \item $((A \cap B) \cup (A \cap C) \cup (B \cap C)) ~\backslash~ (A \cup B \cup C)$.
  \end{itemize}

  \section{[UD] Problem 6.16}
  \begin{enumerate}[(a)]
    \item For every $n$ in $A$, $n = x^2$ where $x$ is an integer, therefore $n$ is an integer, i.e. $n \in B$, so $A \subseteq B$. \hfill $\square$
    \item For every $t$ in $A$, $t$ is a real number, there exists a real number $x=t/2$, such that $t=2x$, so $t \in B$. Therefore $A \subseteq B$. \hfill $\square$
    \item For every point $(x,y)$ in $A$, we have $y=(5-3x)/2$, therefore $2y+3x=5$, which means that $(x,y)$ is also in $B$. So $A \subseteq B$. \hfill $\square$
  \end{enumerate}

  \section{[UD] Problem 6.17}
  \begin{enumerate}[(a)]
    \item $A$ is a proper subset of $B$. For every $(x,y)$ in $A$, we have $xy>0$, so both $x$ and $y$ are nonzero, thus $x^2+y^2>0$, therefore $A$ is a subset of $B$. However, $(1,-1)$ is an element of $B$, but not an element of $A$, so $A$ is a proper subset of $B$. \hfill $\square$
    \item $A$ is a proper subset of $B$. By $theorem 6.10$, we have $A \subseteq$. However, $(0,0)$ is an element of $B$, but not an element of $A$, so $A$ is a proper subset of $B$. \hfill $\square$
  \end{enumerate}

  \section{[UD] Problem 7.1}
  \begin{enumerate}[(a)]
    \item For every $x$ in universe, by definition of complement, if $x \in A$, then $x \notin A^c$ and if $x \notin A^c$ then $x \in (A^c)^c$, therefore we have if $x \in A$, then $x \in (A^c)^c$, i.e. $A$ is a subset of $(A^c)^c$. $(A^c)^c$ is a subset of $A$ likewise. So $(A^c)^c=A$. \hfill $\square$
    \item For every $x$ in $A \cap (B \cup C)$, we have $x \in A$, and $x \in B$ or $C$, so $x \in A$ and $B$ or $x \in A$ and $C$, thus $A \cap (B \cup C)$ is a subset of $(A \cap B) \cup (A \cap C)$. For every $x$ in $(A \cap B) \cup (A \cap C)$, we have $x \in A$ and $B$ or $x \in A$ and $C$, so $x \in A$, and $x \in B$ or $C$, thus $(A \cap B) \cup (A \cap C)$ is a subset of $A \cap (B \cup C)$. So $A \cap (B \cup C) = (A \cap B) \cup (A \cap C)$. \hfill $\square$
    \item For every $x$ in $X ~\backslash~ (A \cap B)$, we have $x \in X$ and, $x \notin A$ or $x \notin B$, thus $x \in X$ and $x \notin A$, or $x \in X$ and $x \notin B$, therefore $X ~\backslash~ (A \cap B)$ is a subset of $(X ~\backslash~ A) \cup (X ~\backslash~ B)$. For every $x$ in $(X ~\backslash~ A) \cup (X ~\backslash~ B)$, we have $x \in X$ and $x \notin A$, or $x \in X$ and $x \notin B$, thus $x \in X$ and, $x \notin A$ or $x \notin B$, so $(X ~\backslash~ A) \cup (X ~\backslash~ B)$ is a subset of $X ~\backslash~ (A \cap B)$. Therefore $X ~\backslash~ (A \cap B) = (X ~\backslash~ A) \cup (X ~\backslash~ B)$. \hfill $\square$
    \item Since $A, B$ are subsets of $X$, for every $x \in X$, if $x \in A$ then $x \in B$ and if $x \notin B$ then $x \notin A$ are equivalent, so $A \subseteq B$ if and only if $(X ~\backslash~ B) \subseteq (X ~\backslash~ A)$. \hfill $\square$
    \item If $A \cap B = B$, then for every $x$, $x \in B$ and $x \in A$ and $B$ are equivalent, so $x \in B$ implies $x \in A$, i.e. $A$ is a subset of $B$. If $B \subseteq A$, for every $x$, $x \in B$ implies $x \in A$, thus $x \in B$ and $x \in A$ and $B$ are equivalent, so $A \cap B = B$. Therefore, $A \cap B = B$ if and only if $B \subseteq A$. \hfill $\square$
  \end{enumerate}

  \section{[UD] Problem 7.8}
  \begin{enumerate}[(a)]
    \item (ii);
    \item (i), (ii), (iii), (iv), (v);
    \item For every $x$ in $(A \cap B)~\backslash~ C$, we have $x \in A$ and $B$ and $x \notin C$, so $x \in A$ and $x \notin C$, and $x \in B$ and $x \notin C$, thus $(A \cap B)~\backslash~ C$ is a subset of $(A~\backslash~ C) \cap ( B~\backslash~ C$. Likewise $(A~\backslash~ C) \cap ( B~\backslash~ C$ is a subset of $(A \cap B)~\backslash~ C$. Therefore $(A \cap B)~\backslash~ C = (A~\backslash~ C) \cap ( B~\backslash~ C)$. \hfill $\square$

  \end{enumerate}
  
  \section{[UD] Problem 7.9}
  \begin{enumerate}[(a)]
    \item For every $x$ in $A ~\backslash~ B$, we have $x \in A$ and $x \notin B$, so $A~\backslash~B$ and $B$ are disjoint. \hfill $\square$
    \item For every $x$ in $A \cup B$, we have $x \in A$ or $x \in B$, so $x \in A$, or $x \in B$ and $x \notin A$, therefore $A \cup B$ is a subset of $(A ~\backslash B) \cup B$. For every $x$ in $(A ~\backslash B) \cup B$, we have $x \in A$, or $x \in B$ and $x \notin A$, so $x \in A$ or $x \in B$, therefore $(A ~\backslash B) \cup B$ is a subset of $A \cup B$. So $A \cup B = (A ~\backslash B) \cup B$ \hfill $\square$
  \end{enumerate}

  \section{[UD] Problem 7.10}
    This statement is false. Here is a counterexample. Let $A=\{1,2\}$, $B=\{1\}$ and $C=\{2\}$, then $A \cup B = A \cup C$, but $B \neq C$. \hfill $\square$

  \section{[UD] Problem 7.11}
    This statement is true. We know that for every $x$, $x \in S$ if and only if ${x} \cap S = {x}$. For every $x \in A$, let $Y = {x}$, then $B \cap Y = A \cap Y = {x}$, so $x \in B$, thus $A$ is a subset of $B$. $B$ is a subset of $A$ likewise. So the statement is true.  \hfill $\square$
\end{document}
